\input ../talk-header.tex
\title{Python}
\subtitle{Introduction}

% If you wish to uncover everything in a step-wise fashion, uncomment
% the following command: 
%\beamerdefaultoverlayspecification{<+->}

\begin{document}

\begin{frame}
  \titlepage
\end{frame}

\begin{frame}{numpy}
  \vphrase{A replacement for MatLab}
\end{frame}

\begin{frame}{numpy}
  \begin{itemize}
  \item \texttt{ndarray} : \textit{an n-dimensional array}
  \end{itemize}

  \texttt{np.array([1,2,3])}\\
  \texttt{np.array([[1, 2], [3, 4]])}\\
  \texttt{np.array([1, 2, 3], dtype = complex)}\\[5mm]

  \texttt{a = np.array(...)}\\
  \texttt{a.shape}\\
  \texttt{a.shape = (n, m)}\\
  \texttt{a.reshape(n, m)}\\[5mm]

  \texttt{a.dtype}\\
  \texttt{a.itemsize}
\end{frame}

\begin{frame}{numpy}
  \texttt{np.empty()}\\
  \texttt{np.empty(shape)}\\[5mm]

  \texttt{np.zeros(shape)}\\
  \texttt{np.ones(shape)}\\[5mm]

  \texttt{np.zeros(48).reshape(6,8)}\\
  \texttt{np.zeros((6,8), dtype=np.int\_)}  
\end{frame}

\begin{frame}{numpy}
  \texttt{np.asarray(a)}\\
  \texttt{np.array(a)}\\
  \texttt{np.fromiter(iterable, dtype, count = -1)}\\[7mm]

  \texttt{np.arange(start, stop, step, dtype)}\\[3mm]
  \texttt{np.linspace(start, stop, num, endpoint, retstep, dtype)}\\[3mm]
  \texttt{np.logspace(start, stop, num, endpoint, base, dtype)}
\end{frame}

\begin{frame}{numpy}
  \texttt{a = arange(30)}\\
  \texttt{s = slice(2,7,2)}\\
  \texttt{a[s]}\\[2mm]
  \texttt{a[2:7:2]}\\[5mm]

  \texttt{m = np.arange(48).reshape(6,8)}\\
  \texttt{m[1:]}\\
  \texttt{m[1:3, 2:5]}\\
  \texttt{m[:, 1]}\\
  \texttt{m[..., 1]}\\
  \texttt{m[:, :-1]}\\
  \texttt{m[1:3, [1,3]]}
\end{frame}

\begin{frame}{numpy}
  \texttt{m = np.arange(48).reshape(6,8)}\\
  \texttt{m[m > 20]}\\[5mm]

  \texttt{n = np.zeros((6,8), dtype=np.int\_)}\\
  \texttt{n[3, 5] = 4}\\
  \texttt{n[5, 3] = 2}\\
  \texttt{m[n > 0]}
\end{frame}

\begin{frame}{numpy}
  Numerical operations are naively element-wise:

  \texttt{a = np.array([1, 2, 3, 4])}\\
  \texttt{b = np.array([10, 20, 30, 40])}\\
  \texttt{a + b}\\
  \texttt{a * b}\\[5mm]

  There are rules (called broadcasting) for how to extend arrays when
  they aren't the same size.  (So be careful if that's not what you
  mean.)
\end{frame}

\begin{frame}{numpy}
  \texttt{a = np.arange(0, 60, 5).reshape(3, 4)}\\
  \texttt{for x in np.nditer(a):}\\
  \texttt{    print(x)}\\[5mm]

  \texttt{for x in np.nditer(a, order='C'):}\\
  \texttt{    print(x)}\\[5mm]

  \texttt{for x in np.nditer(a, order='F'):}\\
  \texttt{    print(x)}\\[5mm]

  \texttt{a.T}\\
  
\end{frame}

\begin{frame}{numpy}
  \texttt{reshape()} : change shape without changing data\\
  \texttt{flat()} : a 1-D iterator over the array\\
  \texttt{flatten()} : return a 1-D copy of the array\\[5mm]

  \texttt{concatenate()} : join arrays along existing axis\\
  \texttt{stack()} : join arrays along new axis\\
  \texttt{hstack()}\\
  \texttt{vstack()}\\[5mm]

  \texttt{split()}\\
  \texttt{hsplit()}\\
  \texttt{vsplit()}
\end{frame}

\begin{frame}{numpy}
  \texttt{$+$, $-$, $/$, $\cdot$}\\[2mm]
  \texttt{np.sin()}, etc.\\[2mm]
  \texttt{np.reciprocal()}\\
  \texttt{np.power()}\\
  \texttt{np.mod()}\\[2mm]
  \texttt{np.real()}, \texttt{np.imag()}, \texttt{np.conj()}, \texttt{np.angle()}\\
\end{frame}

\begin{frame}{numpy}
  \texttt{np.amin()}, \texttt{np.amax()}\\
  \texttt{np.ptp()}\\
  \texttt{np.percentile()}\\
  \texttt{np.median()}\\
  \texttt{np.mean()}\\
  \texttt{np.average()} : weighted average\\[2mm]
  \texttt{np.var()}\\[2mm]
  \texttt{np.sqrt(np.mean(np.abs(x - x.mean()) ** 2))} \\
  \hspace{10mm} $\iff$ \texttt{math.sqrt(np.var())}
\end{frame}

\begin{frame}{numpy}
  \texttt{import numpy.matlib as npmat}\\[2mm]
  \texttt{npmat.empty()}\\
  \texttt{npmat.zeros()}\\
  \texttt{npmat.ones()}\\
  \texttt{npmat.eye())}, \texttt{npmat.identity()}\\
  \texttt{npmat.rand()}\\
  \texttt{npmat.ones()}\\
  \texttt{npmat.ones()}\\
\end{frame}

\begin{frame}{numpy}
  \texttt{import numpy.linalg as npl}\\[2mm]
  \texttt{npl.dot()}
  \texttt{npl.vdot()}
  \texttt{npl.inner()}
  \texttt{npl.matmul()}
  \texttt{npl.determinant()}
  \texttt{npl.solve()}
  \texttt{npl.inv()}
\end{frame}

\begin{frame}{scipy}
  Wrappers on top of classic fortran libraries (LAPACK, etc.).  Built
  on top of \texttt{numpy} data multi-dimensional arrays.
  
  \texttt{import scipy}\\
  \texttt{import scipy.cluster}, etc.

  \begin{itemize}
  \item \texttt{scipy.cluster} : vector quantisation, $k$-means
  \item \texttt{scipy.constants}
  \item \texttt{scipy.fftpack}
  \item \texttt{scipy.integrate}
  \item \texttt{scipy.interpolation}
  \item \texttt{scipy.io}
  \end{itemize}
\end{frame}

\begin{frame}{scipy}
  \begin{itemize}
  \item \texttt{scipy.linalg}
  \item \texttt{scipy.ndimage}
  \item \texttt{scipy.odr} : orthogonal distance regression
  \item \texttt{scipy.optimze}
  \item \texttt{scipy.signal}
  \item \texttt{scipy.sparse}
  \end{itemize}
\end{frame}

\begin{frame}{scipy}
  \begin{itemize}
  \item \texttt{scipy.spatial}
  \item \texttt{scipy.special} : special mathematical functions
  \item \texttt{scipy.stats}
  \end{itemize}
\end{frame}

\begin{frame}{statsmodel}
  \vphrase{\texttt{import statsmodels}}

  \centerline{There are important submodules.}
\end{frame}

\begin{frame}{Retours}
  \begin{itemize}
  \item Qu'est-ce qui vous a plu ?
  \item Qu'est-ce qui vous a manqué ?
  \end{itemize}
\end{frame}


%%%%%%%%%%%%%%%%%%%%%%%%%%%%%%%%%%%%%%%%%%%%%%%%%%%%%%%%%%%%%%%%%%%%%%
%\talksection{Break}

\begin{frame}
  \frametitle{Questions?}
\end{frame}

\end{document}
